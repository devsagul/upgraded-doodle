\usepackage{anysize,epstopdf,float,caption,enumitem,longtable}              

%\marginsize{left}{right}{top}{bottom}:
\marginsize{3cm}{2cm}{2cm}{2cm}
\usepackage{marginnote}

\usepackage{tikz,circuitikz,amsmath}
\usepackage{qrcode}
\usetikzlibrary{arrows,shapes}

\usepackage{array,multirow} % позволяет выравнивать текст в ячейках по вертикали


\usepackage{listings} % Включение кусков кода
\lstset{language=Python, inputencoding=cp1251, breaklines=true, frame=single, showstringspaces=false, basicstyle=\ttfamily}
\lstset{language=HTML, inputencoding=cp1251, breaklines=true, frame=single, showstringspaces=false, basicstyle=\ttfamily}



\renewcommand{\thefigure}{\arabic{figure}}
\renewcommand{\thetable}{\arabic{table}}
\graphicspath{{./img/}}
\usepackage{hyperref}
\usepackage[hypcap]{caption}

% Убрать коробки вокруг гиперссылок
\hypersetup{
  colorlinks   = true, %Colours links instead of ugly boxes
  urlcolor     = red, %Colour for external hyperlinks
  linkcolor    = red, %Colour of internal links
  citecolor   = red %Colour of citations
}

\usepackage{subcaption}

\relpenalty=9999
\binoppenalty=9999

\DeclareCaptionLabelSeparator{tb}{:\\}

\usepackage{tabularx,ragged2e,booktabs,caption}
\newcolumntype{C}[1]{>{\Centering}m{#1}}
\renewcommand\tabularxcolumn[1]{C{#1}}
\captionsetup[table]{justification=raggedleft,
  singlelinecheck=false,labelsep=tb
}

%\usepackage{chngcntr}
%\counterwithin{table}{chapter}
%\counterwithin{figure}{chapter}

\makeatletter
  \def\thesubfigure{\textit{\asbuk{subfigure}}}
  \providecommand\thefigsubsep{}
  \def\p@subfigure{\@nameuse{thefigure}\thefigsubsep}
\makeatother

\newcolumntype{P}[1]{>{\centering\arraybackslash}p{#1}}
\newcolumntype{M}[1]{>{\centering\arraybackslash}m{#1}}
\newcommand*\rot{\rotatebox{90}}

%%% Библиография %%%

\usepackage{cite} % Красивые ссылки на литературу
\makeatletter
\bibliographystyle{bib/utf8gost705u}	% Оформляем библиографию согласно ГОСТ 7.0.5
\renewcommand{\@biblabel}[1]{#1.}	% Заменяем библиографию с квадратных скобок на точку:
\makeatother
\addto{\captionsrussian}{\renewcommand{\bibname}{Список использованных источников}}


%%% Список сокращений %%%
\usepackage[acronym,nopostdot,nonumberlist]{glossaries}
\makeglossaries


\renewcommand{\glossarysection}[2][]{}
%\setlength{\glssymbolwidth}{.1\hsize}
%\setlength{\glsdescwidth}{15cm}





